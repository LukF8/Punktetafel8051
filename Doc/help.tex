\chapter{Konzept} 	% engl. Preface




\noindent
Diese Projektarbeit habe ich während meiner dritten Praxisphase geschrieben.

 


%%%
%Auf die Entstehung des Dikumentes eingehen; Danksagungen; Widmungen und phylisophische Anmerkungen NICHT Übertreiben max. 2 Seiten





%%%%%%%%%%%%%%%%%%%%%%%%%%%%%%%%%%%%%%%%%%%%%%%%%%%%%%%%%%%%%%%%%%%%%%%%
%%%%%%%%%%%%%%%%%%%%%%%%%%%%%%%%%%%%%%%%%%%%%%%%%%%%%%%%%%%%%%%%%%%%%%%%
%CheatSheet V0.01

 


%\begin{comment}
gedruckt werden sollte auf hochwertigem papier min.  80 g/m2


Dies ist \textbf{Version \hgbthesisDate} 
Es gibt Abkürzungen \va 
\emph{Word}
Referenzeren zu label \ref{ch:TechnischeInfos}.
(s.\ Abschn.~\ref{sec:latex}).
(s.\ Kapitel~\ref{sec:latex}).

nicht einrücken des Absatzes
\noindent


Abbildungen Abb.~\ref{fig:CocaColaig}

optionales bindestrich + Anführungszeichen + fußnote 
tat\-säch\-lich "`Paketen"' \footnote{Nicht mehr verfügbar.}
Fußnoten immer mit Punkt abschließen

beidseitig einrücken
\begin{quote}
    
    beidseitig einrücken
\end{quote}

url
\url{www.simeens.com/}

linebreak (böse)
\newline or \\

offline url (ohne resource)
\nolinkurl{.../lange/uurl}


zentrierter text 
\begin{center}%
\begin{tabular}{l}
\nolinkurl{daniel@vogt.ml} \\
Daniel Vogt \\
DHBW Kalsruhe -- Siemens AG\\
Germany
\end{tabular}
\end{center}


italic
\begin{quote}
\verb!\textit{Variable}! $\rightarrow$ \textit{Variable}
\end{quote}



zentrierte box
\begin{center}
\setlength{\fboxrule}{0.2mm}
\setlength{\fboxsep}{2mm}
\fbox{%
\begin{minipage}{0.9\textwidth}
Besonders
\end{minipage}
}
\end{center}


andere box
\begin{center}
\fbox{
    \centering
\begin{minipage}{0.95\textwidth}
\medskip
sth
\medskip
\end{minipage}%
}
\end{center}






großer abstand (statt nur einrücken)
\SuperPar 
Manchmal


drei punkte
\ldots

optionales break ohne bindestrich
(weiter-){\optbreaknh}entwickelt.%

typewriter
\texttt{babel}



\begin{itemize}
\item Produktbezeichnungen \textit{italic}    %schräg
%
\item Markennamen \textsl{salted}   %krass
%
\item Überschriften und Bezeicher \textbf{Boldface}   %fett
%
\item \textsc{Test \emph{Emphasize}}  %richtig schräg
%
\item \underline{Unterstreichungen} %dont use that!   
%
\end{itemize}


Überschriften
\chapter{title}
\section{title}
\subsection{title}
\subsubsection{title}

coole Überschriften (In paragraph)
\paragraph{title}
\subparagraph{title}


mehrzeiliges \verb typewriter listing

Listen
\begin{verbatim}
punkte
\begin{itemize}     ... \end{itemize}
nummern
\begin{enumerate}   ... \end{enumerate}
text
\begin{description} ... \end{description}
\end{verbatim}


Überschrift mit Fußnote
\chapter[Schlussbemerkungen]%
{Schlussbemerkungen%
\protect\footnote{Diese Anmerkung ....}}%


horizontaler platzhalter (tabulator)
\hspace{1em} oder \hspace{1em} lol


Objekt referenzieren
Kapitel ref	\ref{key}
seite referenzieren
Seiten ref		\pageref{key}


Englisch spacing 
\nonfrenchspacing

Gedankenstriche
für englisch---ja
für deutsch -- ja

einfache Anführungszeichen
\glq \grq  (im englischen einfach ganz normal)
normale anführungszeichen
``bla'' "`bla"'


optinal hyphon \-


schlampig setzen  (zeile mit überlenge)
\begin{sloppypar}
Dieser Absatz wird ``schlampig'' (sloppy) gesetzt ...
\end{sloppypar}



Einfache Tabellen \ref{tab:abkuerzungenq}
\begin{table}
\caption{In \texttt{hgb.sty} definierte Abkürzungsmakros.}
\label{tab:abkuerzungenq}
\centering\small
\begin{tabular}{lcccc}
\emph{Element} & 
\texttt{diplom} &
\texttt{master} &
\texttt{bachelor} & 
\texttt{praktikum} 
\\
\hline
\verb!\title! 			& $+$ & $+$ & $+$ & $+$ \\
\hline
\end{tabular}
\end{table}



Tabelle mit leerspalte \ref{tab:abkuerzungen}
\begin{table}
\caption{In \texttt{hgb.sty} definierte Abkürzungsmakros.}
\label{tab:abkuerzungen}
\centering
\begin{tabular}{llp{2cm}ll}
\hline
\verb+\bzw+        & \bzw   & &  \verb+\ua+         & \ua \\
\hline
\end{tabular}
\end{table}


Tabelle flattersatz  (sieht besser aus bei mehr text)

\begin{table}
    \caption{Definitionen der Spalten.}
    \label{tab:apalten} 
    \def\rr{\rightskip=0pt plus1em \spaceskip=.3333em \xspaceskip=.5em\relax}
    \centering\small
    \begin{tabular}{l|p{0.6\textwidth}}
        \emph{Name} & \texttt{Inhalt} 
        \\
        \hline
        \hline
        Something 			& {\rr Long Long long text. Long Long long text. Long Long long text. Long Long long text. Long Long long text. Long Long long text.} \\
        Something 			& {\rr Long Long long text. Long Long long text. Long Long long text. Long Long long text. Long Long long text. Long Long long text.} \\    Something 			& {\rr Long Long long text. Long Long long text. Long Long long text. Long Long long text. Long Long long text. Long Long long text.} \\    Something 			& {\rr Long Long long text. Long Long long text. Long Long long text. Long Long long text. Long Long long text. Long Long long text.} \\
        \hline
    \end{tabular}
\end{table}



Mehr tabelllen optionen
\begin{table}
\caption{Prozessor-Familien im Überblick.}
\label{tab:processors}
\centering
\setlength{\tabcolsep}{5mm}	% separator between columns
\def\arraystretch{1.25}			% vertical stretch factor (Standard = 1.0)
\begin{tabular}{|r||c|c|c|} \hline
& \emph{PowerPC} & \emph{Pentium} & \emph{Athlon} \\
\hline\hline
Manufacturer & Motorola & Intel & AMD \\
\hline
Speed & high & medium & high   \\
\hline
Price & high & high   & medium \\
\hline
\end{tabular}
\end{table}


\begin{comment} 
bild \{fig:CocaCola}
\begin{figure}
\centering
\includegraphics[width=.95\textwidth]{pcs7} %{CS0031}
\caption{Coca-Cola Werbung 1940  }
\label{fig:CocaCola}
\end{figure}
\end{comment}



Caption Platzierung
bei Abbildungen unten
bei Tabellen oben
bei code oben


latex code
\begin{LaTeXCode}
\begin{figure}
    \centering
    \includegraphics[width=.95\textwidth]{pcs7} %{CS0031}
    \caption{Coca-Cola Werbung 1940  }
    \label{fig:CocaCola}
\end{figure}
\end{LaTeXCode}


keine Zeilennummerierung
\verb|\begin{LaTeXCode}[numbers=none]|


Bildquelle~\cite{IBM360}
in caption


\begin{comment} 

Bild Rahmen
\FramePic{\includegraphics[height=50mm]{pcs7}}
\end{comment}




Vektorgrafik Freehand??!! (Programm)

TrueType BaKoMa Fonts Collection (um in bildern schöne schrift zu haben)



%\end{comment}


%\begin{comment}

%Code Listings

\begin{CCode}
	...
\end{CCode}
C++ (ISO):
\begin{CppCode}
	...
\end{CppCode}
Java:
\begin{JavaCode}
	...
\end{JavaCode}
JavaScript:
\begin{JsCode}
	...
\end{JsCode}
PHP:
\begin{PhpCode}
	...
\end{PhpCode}
HTML:
\begin{HtmlCode}
	...
\end{HtmlCode}
CSS:
\begin{CssCode}
	...
\end{CssCode}
XML:
\begin{XmlCode}
	...
\end{XmlCode}
LaTeX:
\begin{LaTeXCode}
	...
\end{LaTeXCode}
Generisch:
\begin{GenericCode}
	...
\end{GenericCode}
%\end{comment}

%\begin{comment}
%In listings kann latex verwendet werden am besten so oder in kommentaren
%Escape with /+\label{ExampleCodeLabel}+/


Nummerierung vortlaaufend (weiter mit der zeilennummer von davor)
\verb|\begin{some Code}[firstnumber=last]|

große code listings
als float element bei > mehrere zeien  (max 2 seiten)
\begin{program}
\caption{Der Titel zu diesem Programmstück.}
\label{prog:xyz}
\begin{JavaCode}
class IrgendWas {
...
}
\end{JavaCode}
\end{program}


listing im anhang
\begin{footnotesize}
\verbatiminput{Projektarbeit2.tex}
\end{footnotesize}


Verweise kapiteln buch
\cite[Kap.\ 5]{Kopka98}

zitat Quellenverweis
volgende passage aus .. \cite{Kopka98}
dann \verb|\begin{quote}| Hier kommt das Zitat



JabRef hilfe für Bibtex    (Programm)






Formeln beispiel
$\bar{a} = f' \cdot \bigl( \frac{f'}{K \cdot u_{\max}} + 1 \bigr)$
$\infty$



zentrieren ohne numerierung  doppel \verb|$|
$$ y = 4 x^2 $$


gleichungen mit nummern
\begin{equation}
f(k) = \frac{1}{N} \sum_{i=0}^{k-1} i^2 . 
\label{eq:MyFirstEquation}
\end{equation}


mehrzeilige gleichungen 
\begin{align}
f_1 (x,y) &= \frac{1}{1-x} + y , \label{eq:f1} \\
f_2 (x,y) &= \frac{1}{1+y} - x , \label{eq:f2}
\end{align}



fallunterscheidungen 
\begin{equation}
f(i) =
\begin{cases}
0             & \text{für $i = 0$},\\
f(i-1) + f(i) & \text{für $i > 0$}.
\end{cases}
\end{equation}



Matrizen

\begin{equation}
\label{eq:f1}
\begin{pmatrix} x' \\ y' \end{pmatrix}
= 
\begin{pmatrix}
\cos \phi & -\sin \phi \\
\sin \phi & \phantom{-}\cos \phi
\end{pmatrix} 
\cdot
\begin{pmatrix}	x \\ y \end{pmatrix} ,
\end{equation}


beachte \phantom{-} für unsichtbare zeichen
alternativ \verb|\bmatrix| für eckige klammern



verwiese auf gleichungen einfach in runden klammern 
(\ref{eq:f1})
oder Gl.~\ref{eq:f1}


alternativ gleichung zentrieren
\begin{center}
$x \in \R$ , $k \in \N_0$, $z = (a + \mathrm{i} \cdot b) \in \Cpx$.
\end{center}

multiplizieren
$\cdot$ und kein * verwenden
fleder angeben
$25 \times 70$


variablen namen mit mehreren zeichen 
\verb|\mathit{variable}|

Beispiel 
\begin{center}
\setlength{\tabcolsep}{4pt}
\begin{tabular}{llll}
\text{Richtig:}  & $\mathit{Scalefactor}^2$ & $\leftarrow$ & \verb!$\mathit{Scalefactor}^2$!
\end{tabular}
\end{center}


Funktionsnamen und Einheiten in Roman schrift
$\mathrm{Distance} $


vor maßeinheiten ein leerzeichen
euro \euro\ text


dezimalzahlen
$3{,}141$


Mathematica zum gleichungen erstellen ???!!!    (Programm)
%\end{comment}


%%Algotihmen Formal definieren
%\begin{comment}
\begin{algorithm}[tbp]
\caption{Bikubische Interpolation in 2D.
$w_{\mathrm{cub}}()$ in Zeile \ref{alg:wcub} bezeichnet die 
eindimensionale kubische Interpolationsfunktion.}
\label{alg:Example}

\begin{algorithmic}[1]% [1] heißt alle Zeilen werden numeriert
\Procedure{BicubicInterpolation}{$I, x, y$} \Comment{$x,y \in \R$}
\Statex Returns the interpolated value of the image $I$ 
at the continuous position $(x, y)$.

\State $\mathit{val} \gets 0$

\For{$j \gets 0, \ldots, 3$} \Comment{iterate over 4 lines}
\State $v \gets \lfloor y \rfloor - 1 + j$
\State $p \gets 0$

\For{$i \gets 0, \ldots, 3$} \Comment{iterate over 4 columns}
\State $u \gets \lfloor x \rfloor - 1 + i$
\State $p \gets p + I(u,v) \cdot w_{\mathrm{cub}}(x - u )$
\label{alg:wcub}
\EndFor

\State $\mathit{val} \gets \mathit{val} + p \cdot w_{\mathrm{cub}}(y - v)$
\EndFor

\State\Return $\mathit{val}$

\EndProcedure
\end{algorithmic}
\end{algorithm}
%\end{comment}

%%%%Quellen



%Kein getrentes quellenverzeichnis
%MakeBibliography[nosplit]{title}  



%Typen von Quellen
%\begin{comment}
	\begin{table}
	\caption{Quellenarten und empfohlene BibTeX-Eintragstypen.}
	\label{tab:QuellenUndEintragstypen}
	\centering
	\begin{tabular}{ll}
	\hline
	Kategorie \emph{literature} & \texttt{@\emph{type}} \\
	\hline
	Buch (Textbuch, Monographie) & \texttt{@book }\\
	Sammelband (Hrsg.\ + mehrere Autoren) & \texttt{@incollection} \\
	Konferenz-, Tagungsband & \texttt{@inproceedings}\\
	Beitrag in Zeitschrift, Journal & \texttt{@article} \\
	Bachelor-, Master-, Diplomarbeit, Dissertation & \texttt{@thesis} \\
	Technischer Bericht, Laborbericht & \texttt{@techreport} \\
	Handbuch, Produktbeschreibung, Norm & \texttt{@manual} \\
	Gesetzestext, Verordnung etc. & \texttt{@misc}\\
	%
	\hline
	Kategorie \emph{avmedia} & \\
	\hline
	Audio (CD) & \texttt{@audio} \\
	Video (auf VHS, DVD, Blu-ray Disk) & \texttt{@video}\\
	Film (Kino) & \texttt{@movie}\\
	Computer Game, Softwareprodukt & \texttt{@software} \\
	%
	\hline
	Kategorie \emph{online} & \\
	\hline
	Webseite, Wiki-Eintrag, Blog etc. & \texttt{@online} \\
	Online-Video, Bild, Grafik & \texttt{@online}\\
	\end{tabular}
	\end{table}
%\end{comment}



%einzelne quellenangabe einfügen (wie sie am ende stehen)
%\printbibliography[keyword=incollectionexample1,heading=noheader]\nocite{Ears99}


%Beipsiele für jeden Typ
\begin{comment}
%buch
@book{BurgerBurge06,
author={Burger, Wilhelm and Burge, Mark},
title={Digitale Bildverarbeitung -- 
Eine Einführung mit Java und ImageJ},
publisher={Springer-Verlag},
address={Heidelberg},
edition={2},
year={2005},
hyphenation={german}

%sammelband  (Abgeshclossener Beitrag in (sammel)buch)
@incollection{Ears99,
author={Burge, Mark and Burger, Wilhelm},
title={Ear Biometrics},
booktitle={Biometrics: Personal Identification in Networked Society},
publisher={Kluwer Academic Publishers},
year={1999},
address={Boston},
editor={Jain, Anil K. and Bolle, Ruud and Pankanti, Sharath},
chapter={13},
pages={273-285},
hyphenation={english}


%Konferenzbeitrag   \texttt{venue} = Tagungsort
@inproceedings{Burger87,
author={Burger, Wilhelm and Bhanu, Bir},
title={Qualitative Motion Understanding},
booktitle={Proceedings of the Intl.\ Joint Conference on Artificial Intelligence},
year={1987},
month={5},
editor={McDermott, John P.},
venue={Mailand},
publisher={Morgan Kaufmann Publishers},
address={San Francisco},
pages={819-821},
hyphenation={english}
}


%Artikel Zeitung Journal Zeitschrift
@article{Mermin89,
author={Mermin, Nathaniel David},
title={What's wrong with these equations?},
journal={Physics Today},
volume={42},
number={10},
year={1989},
pages={9-11},
hyphenation={english}
}


%Bachelor, Master, Diplomarbeiten
%type = phdthesis; mathesis;  Diplomarbeit; Bachelorarbeit
@thesis{Eberl87,
author={Eberl, Gerhard},
title={Automatischer Landeanflug durch Rechnersehen},
type={phdthesis},
year={1987},
month={8},
institution={Universität der Bundeswehr, Fakultät für Raum- und Luftfahrttechnik},
address={München},
url={http://theses.fh-hagenberg.at/thesis/Artner07},
hyphenation={german}
}


%numerisierte berichte; Forschungsberichte; Unternehmen, Projekten; Hochschulinstituten
@techreport{Drake48,
author={Drake, Huber M. and McLaughlin, Milton D. and Goodman, Harold R.},
title={Results obtained during accelerated transonic tests of the {Bell} {XS-1} airplane in flights to a {MACH} number of 0.92},
institution={NASA Dryden Flight Research Center},
year={1948},
month={1},
address={Edwards, CA},
number={NACA-RM-L8A05A},
url={http://www.nasa.gov/centers/dryden/pdf/...05A.pdf},
hyphenation={english}
}


%Ohne Autor Anleitungen; white paper; präsentationen; Produkt Beschriebeungen; Normen
%wichtig doppelklammer beim author
@manual{amsldoc02,
author={{American Mathematical Society}},
title={User's Guide for the {\tt amsmath} Package},
year={2002},
version={2.0},
url={ftp://ftp.ams.org/pub/tex/doc/amsmath/amsldoc.pdf},
hyphenation={english}
}

%Gesetzestexte
@misc{OoeRaumordnungsgesetz94,
title={Oberösterreichisches Raumordnungsgesetz 1994},
howpublished={LGBl 1994/114 idF 1995/93},
url={http://www.ris.bka.gv.at/Dokumente/LrOO/...538.pdf},
hyphenation={german}
}


%Aduio, Video auf DVD CD ... physisch 
@audio{Zappa95,
author={Zappa, Frank},
title={Freak Out},
howpublished={Audio-CD},
year={1995},
month={5},
note={Rykodisc, New York},
hyphenation={english},
keywords={audioexample1}
}



%video physisch
@video{Futurama99,
author={Groening, Matt},
title={Futurama -- Season 1 Collection},
howpublished={DVD},
year={2002},
month={2},
note={Twentieth Century Fox Home Entertainment},
hyphenation={english}
}


%movie
@movie{Nosferatu,
title={Nosferatu -- A Symphony of Horrors},
howpublished={Film},
year={1922},
note={Drehbuch/Regie: F. W. Murnau. Mit Max Schreck, Gustav von Wangenheim, Greta Schröder.},
hyphenation={english}
}


%computer spiele, software
@software{LegendOfZelda98,
author={Miyamoto, Shigeru and Aonuma, Eiji and Koizumi, Yoshiaki},
title={The Legend of Zelda: Ocarina of Time},
howpublished={N64-Spielmodul},
publisher={Nintendo},
year={1998},
hyphenation={english}
}


%%Auf webseiten verweißen
\footnote{\url{http://www.panasonic-broadcast.de}}

%Atribut url ist bei den anderen typen immer zulässig (wenn das Primärpublikum nicht online ist)


%ausßschließlich online verfügbar:
% Wiki-Eintrag, ein digitales Bild,	eine Software, ein YouTube-Video oder ein Blog-Eintrag.

@online{WikiReliquienschrein,
url={http://de.wikipedia.org/wiki/Reliquienschrein},
urldate={2011-06-19}
}

%online video 
@online{HistoryOfComputers2008,
title={History of Computers},
year={2008},
url={http://www.youtube.com/watch?v=LvKxJ3bQRKE},
hyphenation={english}
}

%Bild
@online{IBM360,
url={http://www.plyojump.com/classes/mainframe_era.html}
}

%falls ein usb stick oder cd beigelegt wir ... dann können auch offline links in den quellen verwendet werden \nolinkurl{...} im note feld


%alle Werke auflisten \nocite{*}

 
\end{comment}







%%%%%%%%%%%%%%%%%%%%%%%%%%%%%%%%%%%%%%%%%%%%%%%%%%%%%%%%%%%%%%%%%%%%%%%%
%%%%%%%%%%%%%%%%%%%%%%%%%%%%%%%%%%%%%%%%%%%%%%%%%%%%%%%%%%%%%%%%%%%%%%%%


 




