\chapter{Implementation}
\label{cha:Implementation}


Im folgenden Kapitel wird die Implementation des Programmes beschrieben. Da eine komplette Beschreibung des Codes den Rahmen sprengen würde, werden hier nur ausgewählte Abschnitte des Programmes beschrieben.

Zu Beginn des Programmes werden die beschrieben Speicheradressen festgelegt und den Variablen Werte zugewiesen. Im Anschluss daran wird der Timer initialisiert.

Zu Beginn des Ablauf des Hauptprogrammes werden alle Zahlen der 7 Segmentanzeigen neu gezeichnet. Dieses Zeichnen wird im Folgenden genauer beschrieben. 
\label{code:methode}
 \begin{GenericCode}
zeigeAlleSemente: 
push 00h	;Sichere R0
push 01h	;Sichere R1	
push 02h	;Sichere R2	
mov R0, #00d	;aktuelles segment
mov R1, #00d	;Pointer für wert von segment

mov A, #01d
cpl A
mov R2, A	;selectiertes segment... wird rotiert

;clear all 7segs 
mov P2, #00h
mov P0, #0FFh
mov P1, #0FFh
mov P2, #0FFh 

loopAlleSegmente:
CJNE R0, #08d, doFuerSegment     ;für alle 8 segmente
pop 02h		;Wiederherstellen von R2
pop 01h		;Wiederherstellen von R1
pop 00h		;Wiederherstellen von R0
ret

doFuerSegment:
;	Lade Segment mit entsprechendem Wert
mov p2, R2	;aktivate segment mit P2



mov A, R0	;get wert für
;60h start adresse für die Sement werte s.o.
Add A, #060h	;aktuelles Segment?  
mov R1, A
mov A, @R1

mov dptr, #numbers	;hole richtigs 7Seg code
movc a,@a+dptr
mov p0, a	;gib 7seg code auf P0 aus
mov p1, a	;gib 7seg code auf P1 aus

mov p2, #0FFh	;deaktivte 7Seg
mov p1, #0FFh	;deaktivte 7Seg
mov p0, #0FFh	;deaktivte 7Seg


; Select nächstes Segment 
mov A, R2
RL A
mov R2, A

;continue loop
Inc R0

ljmp loopAlleSegmente
	
\end{GenericCode}

Zu Beginn des Aufrufs von ZeigeAlleSegemente (\ref{code:methode}) werden die Register 0 bis 2 auf dem Stack gespeichert. In die Register 0 und 1 wird 0 geschrieben. Anschließend wird dann der Wert 1 in den Akku geladen. Anschließend wird das Komplementär dieses Wertes gebildet. Somit sind alle Stellen der 8 bit Zahl außer der rechten 1. Die Binär-Zahl ist also 11111110. Dieser Wert wird in das Register 2 geladen. 

Nach dieser Initialisierung findet ein Loop durch die 8 Stellen der Anzeige statt. Dafür wird anfangs verglichen ob schon bereits alle Werte neu geschrieben wurden, indem der Wert von Register 0 mit 8 verglichen wird. R0 speichert also die jeweilige Stelle die adressiert wird. Zu Beginn eines Ablaufes, wird das passende Segment mithilfe des in Register 2 gespeicherten Wertes ausgewählt und so auf 0 gesetzt. Anschließend wird mithilfe des Registers 0 der passenden aktuelle Wert aus dem Speicher ausgewählt. Das passiert in dem der Wert aus R0 um 60 erhöht wird, so adressiert er die passende Speicherstelle. Im Anschluss wird der Wert aus der Datenbank ausgelesen. Anschließend werden alle Anzeigen wieder deaktiviert, damit die Werte neu gezeichnet werden können. 

Am Ende eines Durchlaufes wird der in R2 gespeicherte Wert nach links rotiert, so dass die 0 innerhalb der Zahl um eine Stelle nach links rückt und beim nächsten Mal die passende Zahl in der Anzeige im nächsten Durchlauf ausgewählt wird. Zudem wird der "Counter" Register 0 um eins erhöht.

Nach dem Ablauf des Neu Zeichnens der Zahlen, finden Überprüfungen statt, ob einer der Knöpfe gedrückt worden ist. Dabei werden nur dann Aktionen ausgeführt, wenn der Knopf gedrückt wurde. Eine wichtige Funktion ist dabei die passende Auswahl des Spielstandes. Dabei wird wenn der Knopf gedrückt wurde, folgende Funktion aufgerufen:

 \begin{GenericCode}
	countupPunkte: 	
	push 00h
	push 01h
	cjne R1, #00d, runtercount
	;Count Hoch
	INC @R0
	cjne @R0, #010d, fertigcountPunkte
	mov @R0, #00d
	INC R0
	INC @R0
	cjne @R0, #10d, fertigcountPunkte
	mov @R0, #00d
	DEC R0
	mov @R0, #00d 
	ljmp fertigcountPunkte 
	
	
	runtercount:
	;Count Runter
	DEC @R0
	cjne @R0, #0FFh, fertigcountPunkte
	mov @R0, #09d
	INC R0
	DEC @R0
	cjne @R0, #0FFh, fertigcountPunkte
	mov @R0, #09d
	DEC R0
	mov @R0, #09d  
	
	
	fertigcountPunkte: 
	pop 01h
	pop 00h
	ret
	
\end{GenericCode}

Zu Beginn des Aufrufs der Funktion werden die Werte aus Register 0 und 1 auf den Stack abgelegt. Anschließend wird überprüft, ob der Wert hoch oder runtergezählt werden soll. Dies geschieht durch einen Abgleich mit Register 1 indem vor Aufruf der Funktion der Wert 0 bzw. 1 gespeichert wurde. Abhängig davon wird nach einem Abgleich die passende Methode aufgerufen. Innerhalb der Mehtoden wird dann der Wert auf der Anzeige um eins erhöht. Dabei muss darauf geachtet werden, dass der Wert passend erhöht wird und nicht ein falscher Wert angezeigt wird. Dabei wird beim Herunterzählen geprüft ob die Zehner und Einerstelle passend (nicht größer als 9) ist. Gleichartig  wird beim Runterzählen darauf geachtet, dass der Wert nicht kleiner als 0 wird. 

Neben diesen Überprüfungen gibt es weitere Funktionen, die die Zeit anhalten und alle Werte auf 0 setzen.



